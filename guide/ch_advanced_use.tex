% Status info:
% M. Gates	2006-2009
% A. Wolf	2011-2015
% ArdWar	2012
% B. Gerdes	2013
% TODO insert further additions from wiki?
% Content OK for 0.15+.
% GZ: typo&grammar check 20160329

\chapter{Files and Directories}
\label{sec:FilesAndDirectories}

\section{Directories}
\label{sec:Directories}

Stellarium has many data files containing such things as star catalogue
data, nebula images, button icons, font files and configuration files.
When Stellarium looks for a file, it looks in two places. First, it
looks in the \emph{user directory} for the account which is running
Stellarium. If the file is not found there, Stellarium looks in the
\emph{installation directory}\footnote{The installation directory was
  referred to as the config root directory in previous versions of this
  guide}. Thus it is possible for Stellarium to be installed by an
administrative user and yet have a writable configuration file for
non-administrative users. Another benefit of this method is on
multi-user systems: Stellarium can be installed by the administrator,
and different users can maintain their own configuration and other files
in their personal user accounts.

In addition to the main search path, Stellarium saves some files in
other locations, for example screens shots and recorded scripts.

The locations of the user directory, installation directory,
\emph{screenshot save directory} and \emph{script save directory} vary
according to the operating system and installation options used. The
following sections describe the locations for various operating systems.

\subsection{Windows}
\label{sec:FilesAndDirectories:Windows}

\begin{description}
\item[installation directory] By default this is
  \file{C:\textbackslash{}Program\ Files\textbackslash{}Stellarium\textbackslash{}},
  although this can be adjusted during the installation process.
\item[user directory] This is the Stellarium sub-folder in the
  Application Data folder for the user account which is used to run
  Stellarium. Depending on the version of Windows and its configuration,
  this could be any of the following (each of these is tried, if it
  fails, the next in the list if tried).

\begin{commands}
%APPDATA%\Stellarium\
%USERPROFILE%\Stellarium\
%HOMEDRIVE%\%HOMEPATH%\Stellarium\
%HOME%\Stellarium\
Stellarium's installation directory
\end{commands}

Thus, on a typical Windows Vista/7/10 system with user ``Bob
Dobbs'', the user directory will be:

\begin{commands}
C:\Users\Bob Dobbs\AppData\Roaming\Stellarium\
\end{commands}

The user data directory is unfortunately hidden by default. To make it accessible in the Windows file
explorer, open an \program{Explorer} window and select \menu{Organize... > Folder and
search options}. Make sure folders marked as hidden are now 
displayed. Also, deselect the checkbox to ``hide known file name
endings''.\footnote{This is a very confusing default setting and in fact
  a security risk: Consider you receive an email with some file
  \file{funny.png.exe} attached. Your explorer displays this as
  \file{funny.png}. You double-click it, expecting to open some image
  browser with a funny image. However, you start some unknown program
  instead, and running this \file{.exe} executable program may turn
  out to be anything but funny!}

\item[screenshot save directory] Screenshots will be saved to the
  \file{Pictures/Stellarium} directory, although this can be changed with a command line option (see
  section~\ref{sec:CommandLineOptions}\footnote{Windows Vista users who do not run Stellarium with
    administrator privileges should adjust the shortcut in the start
    menu to specify a different directory for screenshots as the Desktop
    directory is not writable for normal programs. 
    Stellarium includes a GUI option to specify the screenshot
    directory.}).
\end{description}

\subsection{Mac OS X}
\label{sec:FilesAndDirectories:MacOSX}

\begin{description}
\item[installation directory] This is found inside the application
  bundle, \file{Stellarium.app}. See \emph{Inside Application
    Bundles}\footnote{\url{http://www.mactipsandtricks.com/articles/Wiley_HT_appBundles.lasso}}
  for more information.
\item[user directory] This is the sub-directory 
  \file{Library/Preferences/Stellarium/} (or \\
  \file{\textasciitilde{}/Library/Application\ Support/Stellarium} on
  newest versions of Mac OS X) of the user's home
  directory.
\item[screenshot save directory] Screenshots are saved to the user's
  Desktop.
\end{description}

\subsection{Linux}
\label{sec:FilesAndDirectories:Linux}

\begin{description}
\item[installation directory] This is in the
  \file{share/stellarium} sub-directory of the installation prefix,
  i.e., usually \file{/usr/share/stellarium} or
  \file{/usr/local/share/stellarium/}.
\item[user directory] This is the \file{.stellarium} sub-directory of
  user's home directory, i.e.,
  \file{\textasciitilde{}/.stellarium/}. This is a hidden folder, so
  if you are using a graphical file browser, you may want to change
  its settings to ``display hidden folders''.
\item[screenshot save directory] Screenshots are saved to the user's
  home directory.
\end{description}

\section{Directory Structure}
\label{sec:FilesAndDirectories:DirectoryStructure}

Within the \emph{installation directory} and \emph{user directory}
defined in section~\ref{sec:Directories}, files are arranged in the
following sub-directories.

\begin{description}
\item[\file{landscapes/}] contains data files and textures used for
  Stellarium's various landscapes. Each landscape has its own
  sub-directory. The name of this sub-directory is called the
  \emph{landscape ID}, which is used to specify the default landscape in
  the main configuration file, or in script commands.
\item[\file{skycultures/}] contains constellations, common star names and
  constellation artwork for Stellarium's many sky cultures. Each culture
  has its own sub-directory in the skycultures directory.
\item[\file{nebulae/}] contains data and image files for nebula textures.
  In the future Stellarium may be able to support multiple sets of nebula
  images and switch between them at runtime. This feature is not
  implemented for version~\StelVersion, although the directory structure is in
  place - each set of nebula textures has its own sub-directory in the
  nebulae directory.
  %% TODO: Update this if no longer valid.
\item[\file{stars/}] contains Stellarium's star catalogues. In the
  future Stellarium may be able to support multiple star catalogues
  and switch between them at runtime. This feature is not implemented
  for version~\StelVersion, although the directory structure is in
  place -- each star catalogue has its own sub-directory in the stars
  directory.
  %% TODO: Update this if no longer valid. 
\item[\file{data/}] contains miscellaneous data files including fonts,
  solar system data, city locations, etc.
\item[\file{textures/}] contains miscellaneous texture files, such as the
  graphics for the toolbar buttons, planet texture maps, etc.
\item[\file{ephem/}] (optional) may contain data files for planetary
  ephemerides DE430 and DE431 (see~\ref{sec:ExtraData:ephemerides}).
\end{description}

If any file exists in both the installation directory and user
directory, the version in the user directory will be used. Thus it is
possible to override settings which are part of the main Stellarium
installation by copying the relevant file to the user area and modifying
it there.

It is recommended to add new landscapes or sky cultures by creating the relevant files
and directories within the user directory, leaving the installation
directory unchanged. In this manner different users on a multi-user
system can customise Stellarium without affecting the other users. 

\section{The Main Configuration File}
\label{sec:ConfigurationFile}

The main configuration file is read each time Stellarium starts, and
settings such as the observer's location and display preferences are
taken from it. Ideally this mechanism should be totally transparent to
the user -- anything that is configurable should be configured ``in''
the program GUI. However, at time of writing Stellarium isn't quite
complete in this respect, despite improvements in each version. Some
settings, esp.\ color values for lines, grids, etc.\ can only be
changed by directly editing the configuration file.\footnote{Color
  values can be edited interactively by the Text User Interface plugin
  (see~\ref{sec:plugins:TextUserInterface}).} This section describes
some of the settings a user may wish to modify in this way, and how to
do it.

If the configuration file does not exist in the \emph{user directory}
when Stellarium is started (e.g., the first time the user starts the
program), one will be created with default values for all settings
(refer to section~\ref{sec:FilesAndDirectories} Files and
Directories for the location of the user directory on your operating
system). The name of the configuration file is
\file{config.ini}\footnote{It is possible to specify a different name
  for the main configuration file using the \texttt{-\/-config-file}
  command line option. See section~\ref{sec:CommandLineOptions} Command 
  Line Options for details.}.

The configuration file is a regular text file, so all you need to edit
it is a text editor like \program{Notepad} on Windows, \program{Text Edit} on
the Mac, or \program{nano}/\program{vi}/\program{gedit}/\program{emacs}/\program{leafpad} etc.\ on Linux.

%The following sub-sections contain details on how to make commonly used
%modifications to the configuration file. 
A complete list of configuration file options and values may be found
in appendix~\ref{sec:config.ini} Configuration File.



\section{Getting Extra Data}
\label{sec:ExtraData}

Stellarium is packaged with over 600 thousand stars in the normal
program download, but much larger star catalogues may be downloaded
in the \emph{Tools} tab of the \emph{Configuration} dialog.


\subsection{Alternative Planet Ephemerides: DE430, DE431}
\label{sec:ExtraData:ephemerides}

\newFeature{0.15}
By default, Stellarium uses the \indexterm{VSOP87} planetary theory,
an analytical solution which is able to deliver planetary positions
for any input date. However, its use is recommended only for the year
range $-4000\ldots+8000$. Outside this range, it seems to be usable
for a few more millennia without too great errors, but with degrading accuracy. 

Since V0.15 you can install extra data files which allow access to the
numerical integration runs \indexterm{DE430} and \indexterm{DE431}
from NASA's Jet Propulsion Laboratory (JPL). The data files have to be
downloaded separately, and most users will likely not need them. DE430
provides highly accurate data for the years $+1550\ldots+2650$, while
DE431 covers years $-13000\ldots+17000$, which allows e.g.\ 
archaeoastronomical research on Mesolithic landscapes. Outside these
year ranges, positional computation falls back to VSOP87.

To enable use of these data, download the files from
JPL\footnote{\url{ftp://ssd.jpl.nasa.gov/pub/eph/planets/Linux/}. (Also
  download from this directory if you are not running Linux!)}:

\begin{tabular}{ccc}
\toprule
\emph{Ephemeris}&\emph{Filename}& \emph{MD5 hash}\\\midrule
DE430& \file{linux\_p1550p2650.430} &\texttt{707c4262533d52d59abaaaa5e69c5738}\\
DE431& \file{lnxm13000p17000.431}   &\texttt{fad0f432ae18c330f9e14915fbf8960a}\\\bottomrule
\end{tabular}


The files can be placed in a folder named \file{ephem} inside either
the \emph{installation directory} or the \emph{user directory}
(see \ref{sec:FilesAndDirectories:DirectoryStructure}). Alternatively,
if you have them already stored elsewhere, you may add the path to
\file{config.ini} like:

\begin{configfile}
[astro]
de430_path = C:/Astrodata/JPL_DE43x/linux_p1550p2650.430
de431_path = C:/Astrodata/JPL_DE43x/lnxm13000p17000.431
\end{configfile}

For fast access avoid storing them on a network drive or USB pendrive!

You activate use of either ephemeris in the configuration panel
(\key{F2}). If you activate both, preference will be given for DE430
if the simulation time allows it. Outside of the valid times, VSOP87
will always be used.



\chapter{Command Line Options}
%\label{command-line-options}
\label{sec:CommandLineOptions}

Stellarium's behaviour can be modified by providing parameters to the
program when it is called via the command line. See table for a full list:

\begin{longtabu} to \textwidth {l|l|X}\toprule
\emph{Option}         & \emph{Option Parameter} & \emph{Description}\\\midrule
-\/-help or -h        & {[}none{]}       & Print a quick command line help message, and exit. \\\midrule
-\/-version or -v     & {[}none{]}       & Print the program name and version information, and exit. \\\midrule
-\/-config-file or -c & config file name & Specify the configuration file name. The default value is \file{config.ini}.

The parameter can be a full path (which will be used verbatim) or a partial path.

Partial paths will be searched for inside the regular search paths
unless they start with a ``\file{.}'', which may be used to explicitly
specify a file in the current directory or similar.

For example, using the option \texttt{-c\ my\_config.ini} would resolve to the file 
\file{\textless{}user\ directory\textgreater{}/my\_config.ini} whereas 
\file{-c\ ./my\_config.ini} can be used to explicitly say the file
\file{my\_config.ini} in the current working directory.\\\midrule
-\/-restore-defaults &  [none]    & Stellarium will start with the default configuration. 
                                    Note: The old configuration file will be overwritten. \\\midrule
-\/-user-dir         & path       & Specify the user data directory. \\\midrule
-\/-screenshot-dir   & path       & Specify the directory to which screenshots will be saved. \\\midrule
-\/-full-screen      & yes or no  & Over-rides the full screen setting in the config file. \\\midrule
-\/-home-planet      & planet     & Specify observer planet (English name). \\\midrule
-\/-altitude         & altitude   & Specify observer altitude in meters. \\\midrule
-\/-longitude        & longitude  & Specify latitude, e.g. +53d58'16.65" \\\midrule
-\/-latitude         & latitude   & Specify longitude, e.g. -1d4'27.48" \\\midrule
-\/-list-landscapes  & {[}none{]} & Print a list of available landscape IDs and exit. \\\midrule
-\/-landscape        & landscape ID & Start using landscape whose ID matches the passed parameter (dir name of landscape). \\\midrule
-\/-sky-date         & date       & The initial date in \texttt{yyyymmdd} format. \\\midrule
-\/-sky-time         & time       & The initial time in \texttt{hh:mm:ss} format. \\\midrule
-\/-startup-script   & script name & The name of a script to run after the program has started. \\\midrule
-\/-fov              & angle      & The initial field of view in degrees. \\\midrule
-\/-projection-type  & ptype      & The initial projection type (e.g. \texttt{perspective}). \\\midrule
-\/-dump-opengl-details or -d     & {[}none{]} & Dump information about OpenGL support to logfile. 
                                                 Use this is you have graphics problems and want to send a bug report. \\\midrule
-\/-angle-mode or -a & {[}none{]} & Use ANGLE as OpenGL ES2 rendering engine (autodetect Direct3D version).\footnote{On Windows only}\\\midrule
-\/-angle-d3d9 or -9 & {[}none{]} & Force use Direct3D 9 for ANGLE OpenGL ES2 rendering engine.\footnotemark[1]\\\midrule
-\/-angle-d3d11      & {[}none{]} & Force use Direct3D 11 for ANGLE OpenGL ES2 rendering engine.\footnotemark[1]\\\midrule
-\/-angle-warp       & {[}none{]} & Force use the Direct3D 11 software rasterizer for ANGLE OpenGL ES2 rendering engine.\footnotemark[1]\\\midrule
-\/-mesa-mode or -m  & {[}none{]} & Use MESA as software OpenGL rendering engine.\footnotemark[1]\\\midrule
-\/-safe-mode or -s  & {[}none{]} & Synonymous to -\/-mesa-mode.\footnotemark[1]\\\midrule
-\/-fix-text or -t   & {[}none{]} & Alternative way of creating the Info text, required on some systems.\footnote{E.g., Raspberry Pi 2 with Raspbian Jessie and VC4 drivers from February 2016. A bugfix should be available later in 2016.}\\\bottomrule
\end{longtabu}

\noindent \newFeature{0.15} If you want to avoid adding the same
switch every time when you start Stellarium from the command line, you
can also set an environment variable \file{STEL\_OPTS} with your
default options. 

\section{Examples}
\label{sec:CommandLineOptions:Examples}

\begin{itemize}
\item To start Stellarium using the configuration file,
  \file{configuration\_one.ini} situated in the user directory (use either of
  these):

\begin{commands}
stellarium --config-file=configuration_one.ini
stellarium -c configuration_one.ini
\end{commands}

\item To list the available landscapes, and then start using the
  landscape with the ID ``ocean''
\begin{commands}
stellarium --list-landscapes 
stellarium --landscape=ocean
\end{commands}
\end{itemize}

%% GZ found in 2015:
\noindent Note that console output (like \command{-\/-list-landscapes}) on Windows is not possible. 

%%% Local Variables: 
%%% mode: latex
%%% TeX-master: "guide"
%%% End: 

