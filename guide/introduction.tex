% Status info:
% M. Gates	2006-2009
% A. Wolf	2011-2015
% ArdWar	2012
% B. Gerdes	2013
% Additions inserted from wiki 2015-12-26
% Content OK for 0.14+.
% TODO: add historical notes
% TODO: typo&grammar check


\chapter{Introduction}
\label{ch:Introduction}

\emph{Stellarium} is a software project that allows people to use their
home computer as a virtual planetarium. It calculates the positions of
the Sun and Moon, planets and stars, and draws how the sky would look to
an observer depending on their location and the time. It can also draw
the constellations and simulate astronomical phenomena such as meteor
showers, and solar or lunar eclipses.

Stellarium may be used as an educational tool for teaching about the
night sky, as an observational aid for amateur astronomers wishing to
plan a night's observing, or simply as a curiosity (it's fun!). Because
of the high quality of the graphics that Stellarium produces, it is used
in some real planetarium projector products. Some amateur astronomy
groups use it to create sky maps for describing regions of the sky in
articles for newsletters and magazines.

The development of a powerful scripting system has been continuing for
a number of years now and can now be called operational. The use of a
script was recognised as a perfect way of arranging a display of a
sequence of astronomical events from the earliest versions of
Stellarium and a simple system called \emph{Stratoscript} was
implemented. The scipting facility is Stellarium's version of a
\emph{Presentation}, a feature that may be used to run an astronomical
or other presentation for instruction or entertainment from within the
Stellarium program. The original \emph{Stratoscript} was quite limited in
what it could do so a new Stellarium Scripting System has been
developed. As of version 0.14.0 a new scripting engine 
% is still in development but it 
has reached a level where it 
has all required features for usage, however new
commands may be added from time to time. Since version 0.14.0 support
of scripts for the \emph{Stratoscript} engine has been discontinued.

Stellarium is still under  development, and by the time you read
this guide, a newer version may have been released with even more
features that those documented here. Check for updates to Stellarium at
the Stellarium website.

If you have questions and/or comments about this guide, 
% please email the author. %% GZ This is IMHO no longer valid.
% For comments 
or about Stellarium itself, visit the Stellarium forums\footnote{%
\url{https://sourceforge.net/p/stellarium/discussion/278769/}}.


\section{Historical notes}
\label{sec:Introduction:HistoricalNotes}

%% GZ: I put meta-comments under %% and outlines of things which should be included but which I cannot write under single comment signs (%).

%% TODO: A short history. Please one of the original team, this would be really useful to fill in!!!
%% It may also be possible, if everyone from the original team adds 1/2-1 page under his name, to have some nice anecdotes?

Stellarium has been developed by a small team of enthusiastic developers. 
%% Please replace the following sentence by something corrected!
[We should have a short description here!]
% Fabien Chereau started the project in 2000,%?
% and soon found support by his brother Guillaume, Matthew Gates,
% Timothy Reaves, Johannes Gajdosik, Barry Gerdes, and Alexander Wolf.
% The artist Johan Meuris contributed the graphical elements, not only
% the GUI button icons, but a full set of drawings of our Western 88
% constellations.

% It may be interesting for "software archaeologists" to dig into very old versions still available on SourceForge... 
%% MAYBE SOME DEV. HIGHLIGHTS? Fabien?

%% AND OTHERS?
%% MAYBE PRESENT THE CONTRIBUTIONS OF THEM? 
% In 200x, Guillaume ported Stellarium first to the Nokia Maemo platform, then to Android. 
% The Android version differs a bit from the desktop version: 
%% WHY, IN WHAT PARTS? Can this guide be applied/extended to the Android version?

By 2010 the team was preparing a first ``final'' release 1.0. 
%% WHAT HAPPENED THEN?

Unfortunately time is evolving, and most members of the original
development team are no longer able to devote most of their spare time
to the project. Fortunately some are still available for limited work
which requires specific knowledge about the project. Currently (2016)
most maintenance and regular releases are created by Alexander
Wolf. Georg Zotti joined around 2010 and is working on extensions of Stellarium's applicability
mostly in the fields of historical and cultural astronomy (which means Stellarium is getting more accurate), but also on
graphic items like the comet tails or the Zodiacal Light.

%% Whoever feels left out, please present yourself!


A great leap forward was the 0.13 series. We had to adapt to
evolutions in the Qt framework which Stellarium is based
on. Fortunately Fabien and Guillaume had enough insight into the
framework to do this. This also means the series 0.13 and later
require more modern hardware than earlier versions. Computers 
with dedicated 3D graphics cards from 2008 and later, and 
most computers from 2011 and later are able to run it. 

Stellarium has been kindly supported by ESA in their Summer of Code in
Space initiatives, which resulted in better planetary rendering
(2012), and the web-based remote control and an alternative solution
for planetary positions based on the DE430/DE431 ephemeris (2015).


This guide is based on the user guide written by Matthew Gates for
version 0.10 around 2008, and updated by Barry Gerdes up to version
0.12. %% PLEASE CORRECT ME!
The user documentation has been developed on the Stellarium wiki for
some time, but we (Alexander and Georg) feel that a single book may
be the better format for offline reading. The PDF version of this 
guide has a clickable table of contents and clickable hyperlinks.

This new edition of the guide will not contain notes about using
earlier versions than 0.13 or using very outdated hardware. Some
references to previous version may still be made for completeness, 
but if you are using earlier versions
for particular reasons, please use the older guides.

%%% Local Variables: 
%%% mode: latex
%%% TeX-master: "guide"
%%% End: 
