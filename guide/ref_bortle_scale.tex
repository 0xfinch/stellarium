% Status info:
% M. Gates	2006-2009
% A. Wolf	2011-2015
% ArdWar	2012
% B. Gerdes	2013
% TODO insert further additions from wiki?
% GZ Content OK for 0.15+.
% 20160329 GZ: typo&grammar check 
% 20160407 GZ: Split Bortle Scale into chapter for reference appendix

\chapter{The Bortle Scale of Light Pollution}
\label{ch:BortleScale}

In Sky\&Telescope February 2001 \name{John E.\ Bortle} published a sky
quality scale which describes the amount of light
pollution. Stellarium's light pollution setting tries to follow this
scale. For completeness we reproduce from Wikipedia: 


\section{Excellent dark sky site}
\textbf{Level:} 1 \\
\textbf{Limiting magnitude (eye):} 7.6 -- 8.0

Zodiacal light, gegenschein, zodiacal band visible; M33 direct vision
naked-eye object; Scorpius and Sagittarius regions of the Milky Way
cast obvious shadows on the ground; Airglow is readily visible;
Jupiter and Venus affect dark adaptation; surroundings basically
invisible.

\section{Typical truly dark site}
\textbf{Level:} 2 \\
\textbf{Limiting magnitude (eye):} 7.1 -- 7.5

Airglow weakly visible near horizon; M33 easily seen with naked eye;
highly structured Summer Milky Way; distinctly yellowish zodiacal
light bright enough to cast shadows at dusk and dawn; clouds only
visible as dark holes; surroundings still only barely visible
silhouetted against the sky; many Messier globular clusters still
distinct naked-eye objects.

\section{Rural sky}
\textbf{Level:} 3 \\
\textbf{Limiting magnitude (eye):} 6.6 -- 7.0

Some light pollution evident at the horizon; clouds illuminated near
horizon, dark overhead; Milky Way still appears complex; M15, M4, M5,
M22 distinct naked-eye objects; M33 easily visible with averted
vision; zodiacal light striking in spring and autumn, color still
visible; nearer surroundings vaguely visible.

\section{Rural/suburban transition}
\textbf{Level:} 4 \\
\textbf{Limiting magnitude (eye):} 6.1 -- 6.5

Light pollution domes visible in various directions over the horizon;
zodiacal light is still visible, but not even halfway extending to the
zenith at dusk or dawn; Milky Way above the horizon still impressive,
but lacks most of the finer details; M33 a difficult averted vision
object, only visible when higher than 55$\degree$; clouds illuminated
in the directions of the light sources, but still dark overhead;
surroundings clearly visible, even at a distance.

\section{Suburban sky}
\textbf{Level:} 5 \\
\textbf{Limiting magnitude (eye):} 5.6 -- 6.0

Only hints of zodiacal light are seen on the best nights in autumn and
spring; Milky Way is very weak or invisible near the horizon and looks
washed out overhead; light sources visible in most, if not all,
directions; clouds are noticeably brighter than the sky.

\section{Bright suburban sky}
\textbf{Level:} 6 \\
\textbf{Limiting magnitude (eye):} 5.1 -- 5.5

Zodiacal light is invisible; Milky Way only visible near the zenith;
sky within 35$\degree$ from the horizon glows grayish white; clouds
anywhere in the sky appear fairly bright; surroundings easily visible;
M33 is impossible to see without at least binoculars, M31 is modestly
apparent to the unaided eye.

\section{Suburban/urban transition}
\textbf{Level:} 7 \\
\textbf{Limiting magnitude (eye):} 5.0 at best

Entire sky has a grayish-white hue; strong light sources evident in
all directions; Milky Way invisible; M31 and M44 may be glimpsed with
the naked eye, but are very indistinct; clouds are brightly lit; even
in moderate-sized telescopes the brightest Messier objects are only
ghosts of their true selves.

\section{City sky}
\textbf{Level:} 8 \\
\textbf{Limiting magnitude (eye):} 4.5 at best

Sky glows white or orange --- you can easily read; M31 and M44 are
barely glimpsed by an experienced observer on good nights; even with
telescope, only bright Messier objects can be detected; stars forming
familiar constellation patterns may be weak or completely invisible.

\section{Inner City sky}
\textbf{Level:} 9 \\
\textbf{Limiting magnitude (eye):} 4.0 at best

Sky is brilliantly lit with many stars forming constellations
invisible and many weaker constellations invisible; aside from
Pleiades, no Messier object is visible to the naked eye; only objects
to provide fairly pleasant views are the Moon, the Planets and a few
of the brightest star clusters.


%%% Local Variables: 
%%% mode: latex
%%% TeX-master: "guide"
%%% End: 

