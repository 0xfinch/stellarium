% Status info:
% M. Gates	2006-2009
% A. Wolf	2011-2015
% ArdWar	2012
% B. Gerdes	2013
% Additions inserted from wiki 2015-12-26
% Content OK for 0.14+.
% 2016-07: Historical notes added by Fabien.
% TODO: typo&grammar check


\chapter{Introduction}
\label{ch:Introduction}

\emph{Stellarium} is a software project that allows people to use their
home computer as a virtual planetarium. It calculates the positions of
the Sun and Moon, planets and stars, and draws how the sky would look to
an observer depending on their location and the time. It can also draw
the constellations and simulate astronomical phenomena such as meteor
showers or comets, and solar or lunar eclipses.

Stellarium may be used as an educational tool for teaching about the
night sky, as an observational aid for amateur astronomers wishing to
plan a night's observing or even drive their telescopes to observing
targets, or simply as a curiosity (it's fun!). Because of the high
quality of the graphics that Stellarium produces, it is used in some
real planetarium projector products and museum projection setups. Some
amateur astronomy groups use it to create sky maps for describing
regions of the sky in articles for newsletters and magazines, and the
exchangeable sky cultures feature invites its use in the field of
Cultural Astronomy research and outreach.

Stellarium is still under development, and by the time you read
this guide, a newer version may have been released with even more
features than those documented here. Check for updates to Stellarium at
the Stellarium website\footnote{\url{http://stellarium.org}}.

If you have questions and/or comments about this guide, or about
Stellarium itself, visit the Stellarium site at
LaunchPad\footnote{\url{https://launchpad.net/stellarium}} or our
SourceForge
forums\footnote{\url{https://sourceforge.net/p/stellarium/discussion/278769/}}.


\section{Historical notes}
\label{sec:Introduction:HistoricalNotes}

Fabien Ch\'ereau started the project during the summer 2000, and throughout
the years found continuous support by a small team of enthusiastic developers.

Here is a list of past and present major contributors sorted roughly by date of
arrival on the project:
\newpage
\begin{description}
\item[Fabien Ch\'ereau] original creator, maintainer, general development
\item[Matthew Gates] maintainer, original user guide, user support, general development
\item[Johannes Gajdosik] astronomical computations, large star catalogs support
\item[Johan Meuris] GUI design, website creation, drawings of our 88 Western constellations
\item[Nigel Kerr] Mac OSX port
\item[Rob Spearman] funding for planetarium support
\item[Barry Gerdes] user support, tester, Windows support. Barry
  passed away in October 2014 at age 80. He was a major contributor on
  the forums, wiki pages and mailing list where his good will and
  enthusiasm is strongly missed. \ifthenelse{\equal{\StelVersion}{0.15.0}}{Version 0.15 of Stellarium is
  dedicated in his memory.}{} RIP Barry.
\item[Timothy Reaves] ocular plugin
\item[Bogdan Marinov] GUI, telescope control, other plugins
\item[Diego Marcos] SVMT plugin
\item[Guillaume Ch\'ereau] display, optimization, Qt upgrades
\item[Alexander Wolf] maintainer, DSO catalogs, user guide, general development
\item[Georg Zotti] astronomical computations, Scenery 3D and ArchaeoLines plugins, general development, user guide, user support
\item[Marcos Cardinot] MeteorShowers plugin
\item[Florian Schaukowitsch] Scenery 3D plugin, Remote Control plugin, Qt/OpenGL internals
\end{description}

Unfortunately time is evolving, and most members of the original
development team are no longer able to devote most of their spare time
to the project (some are still available for limited work
which requires specific knowledge about the project).

As of 2017, the project's maintainer is Alexander Wolf, doing most maintenance
and regular releases. Major new features are contributed mostly by Georg Zotti
and his team focussing on extensions of Stellarium's applicability in the fields of
historical and cultural astronomy research (which means Stellarium is getting more
accurate), but also on graphic items like comet tails or the Zodiacal Light.

\vspace{1\baselineskip}
\noindent A detailed track of development can be found in the
\file{ChangeLog} file in the installation folder. A few important
milestones for the project:
\begin{description}
\item[2000] first lines of code for the project
\item[2001-06] first public mention (and users feedbacks!) of the
  software on the French newsgroup \texttt{fr.sci.astronomie.amateur} 
  \footnote{\url{https://groups.google.com/d/topic/fr.sci.astronomie.amateur/OT7K8yogRlI/discussion}}
\item[2003-01] Stellarium reviewed by Astronomy magazine
\item[2003-07] funding for developing planetarium features (fisheye projection and other features)
\item[2005-12] use accurate (and fast) planetary model
\item[2006-05] Stellarium ``Project Of the Month'' on SourceForge
\item[2006-08] large stars catalogs
\item[2007-01] funding by ESO for development of professional astronomy extensions (VirGO)
\item[2007-04] developers' meeting near Munich, Germany
\item[2007-05] switch to the Qt library as main GUI and general purpose library
\item[2009-09] plugin system, enabling a lot of new development
\item[2010-07] Stellarium ported on Maemo mobile device
\item[2010-11] artificial satellites plugin
\item[2014-06] high quality satellites and Saturn rings shadows, normal mapping for moon craters
\item[2014-07] V0.13: adapt to OpenGL evolutions in the Qt framework, now requires more modern graphic hardware than earlier versions
\item[2015-04] V0.13.3: Scenery 3D plugin
\item[2015-10] V0.14.0: Accurate precession
\item[2016-07] V0.15: Remote Control plugin
\item[2016-12] V0.15.1: DE430\&DE431, AstroCalc, DSS layer, and Stellarium acting as SpoutSender
\end{description}

Stellarium has been kindly supported by ESA in their Summer of Code in
Space initiatives, which resulted in better planetary rendering
(2012), the Meteor Showers plugin (2013) and the web-based remote
control and an alternative solution for planetary positions based on
the DE430/DE431 ephemeris (2015).


This guide is based on the user guide written by Matthew Gates for
version 0.10 around 2008. The guide was then ported to the Stellarium
wiki and continuously updated by Barry Gerdes and Alexander Wolf up to version 0.12. 

The user documentation has been developed further on the Stellarium
wiki for some time, but without Barry started to fall out of sync with
the actual program.  We (Alexander and Georg) have ported the texts
back to \LaTeX{} and updated and added information where necessary. We
feel now that a single book may be the better format for offline
reading. The PDF version of this guide has a clickable table of
contents and clickable hyperlinks.

This new edition of the guide will not contain notes about using
earlier versions than 0.13 or using very outdated hardware. Some
references to previous version may still be made for completeness, 
but if you are using earlier versions
for particular reasons, please use the older guides.

%%% Local Variables: 
%%% mode: latex
%%% TeX-PDF-mode: t
%%% TeX-master: "guide"
%%% End: 
