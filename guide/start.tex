% Status info:
% M. Gates	2006-2009
% A. Wolf	2011-2015
% Lee Carré	2011
% ArdWar	2012
% MisterE	2013
% B. Gerdes	2013
% G. Zotti	2014-2015
% Additions inserted from wiki 2015-12-26
% GZ checked grammar and structure, added ANGLE details and Troubleshooting.
% Content OK for 0.15+.


\chapter{Getting Started}
\label{ch:GettingStarted}

\section{System Requirements}\index{System Requirements}
\label{sec:GettingStarted:SystemRequirements}

\subsection{Minimum}
\begin{itemize}
\item Linux/Unix; Windows 7 and later (It may run on Vista, but unsupported. A special version for XP is still available); OS X 10.7.4 and later
\item 3D graphics card which supports OpenGL 3.0 and GLSL 1.3
\item 512 MB RAM
\item 250 MB free on disk
\end{itemize}

\subsection{Recommended}
\begin{itemize}
\item Linux/Unix; Windows 7 and later; OS X 10.8.5 and later
\item 3D graphics card which supports OpenGL 3.3 and above and GLSL1.3 and later
\item 1 GB RAM or more
\item 1.5 GB free on disk. (About 3GB extra required for the optional DE431 files.)
\end{itemize}
 A dark room for realistic rendering --- details like the Milky Way, Zodiacal Light or
star twinkling can't be seen in a bright room.

\section{Downloading}\index{Downloading}
\label{sec:GettingStarted:Downloading}

Download the correct package for your operating system directly from the main page,
\url{http://stellarium.org}.

\section{Installation}\index{Installation}
\label{sec:GettingStarted:Installation}

\subsection{Windows}\index{Windows}

\begin{enumerate}
\item Double click on the installer file you downloaded:
\begin{itemize}
\item \file{stellarium-\StelVersion-win64.exe}
\item \file{stellarium-\StelVersion-win32.exe} on Windows x86
\item \file{stellarium-\StelVersion-classic-win32.exe} for XP
\end{itemize}
\item Follow the on-screen instructions.
\end{enumerate}

\subsection{OS X}\index{OS X}

\begin{enumerate}
\item
  Locate the \file{Stellarium-\StelVersion.dmg} file in
  Finder and double click on it or open it using the Disk Utility
  application. Now, a new disk appears on your desktop and Stellarium is
  in it.
\item
  Open the new disk and please take a moment to read the \file{ReadMe} file.
  Then drag \file{Stellarium} to the Applications folder.
\item
  Note: You should copy Stellarium to the Applications folder before
  running it --- some users have reported problems running it directly
  from the disk image (\file{.dmg}).
\end{enumerate}

\subsection{Linux}\index{Linux}

Check if your distribution has a package for Stellarium already --- if
so you're probably best off using it. If not, you can download and build
the source.

For Ubuntu we provide a package repository with the latest stable
releases. Open a terminal and type:

\begin{commands}
sudo add-apt-repository ppa:stellarium/stellarium-releases
sudo apt-get update
sudo apt-get install stellarium
\end{commands}


\section{Running Stellarium}\index{Running Stellarium}
\label{sec:GettingStarted:Running}

\subsection{Windows}\index{Windows}
\label{sec:GettingStarted:Running:Windows}

The Stellarium installer creates a whole list of items in the
\textbf{Start Menu} under the \textbf{Programs/Stellarium}
section. Select one of these to run Stellarium:
\begin{description}
  \item[Stellarium] OpenGL version. This is the most efficient for
    modern PCs and should be used when you have installed appropriate
    OpenGL drivers.
  \item[Stellarium (ANGLE mode)] Uses Direct3D translation of the
    OpenGL rendering via ANGLE library. Lets the system decide which
    version of Direct3D is available. In case the system reports
    support for Direct3D~11 but if you experience odd effects like
    missing buttons or planets, directly use
  \item[Stellarium (ANGLE Direct3D 9 mode)] Uses Direct3D translation
    of the OpenGL rendering via ANGLE library. Forces Direct3D
    version~9. \emph{This should be used if you don't see buttons or have
    trouble with other ANGLE modes.}
  \item[Stellarium (ANGLE Direct3D 11 mode)] Uses Direct3D translation
    of the OpenGL rendering via ANGLE library. Forces Direct3D
    version~11. Some systems don't seem to work properly with this.
  \item[Stellarium (ANGLE WARP mode)] Uses DirectX3D~11 software rendering via ANGLE
    library. This should work on any PC without dedicated graphics
    card. However on many systems this fails, it is unclear why.
  \item[Stellarium (MESA mode)] Uses software rendering via MESA
    library. This should work on any PC without dedicated graphics
    card. However on some systems this also fails, it is unclear
    why\footnote{This was the emergency fallback solution for the 0.13
      series. We have reports that 0.13.2-MESA works on a system where
      0.14 does not.}
\end{description}
  On startup, a diagnostic check is performed to test whether the
  graphics hardware is capable of running. If all is fine, you will
  see nothing of it.  Else you may see an error panel informing you
  that your computer is not capable of running Stellarium (``No
  OpenGL~2 found''), or that there is only OpenGL~2.1 support. The
  latter means you will be able to see some graphics, but depending on
  the type of issue you will have some bad graphics. For example, on
  an Intel GMA4500 there is only a minor issue in Night Mode, while on
  other systems we had reports of missing planets. If you see this,
  try running in Direct3D~9 or MESA mode, or upgrade your system.

  When you have found a mode that works on your system, you can delete
  the other links.

\subsection{OS X}\index{OS X}
\label{sec:GettingStarted:Running:MacOSX}

Double click on the \emph{Stellarium} application.  Add it to your
\textbf{Dock} for quick access.

\subsection{Linux}\index{Linux}
\label{sec:GettingStarted:Running:Linux}

If your distribution had a package you'll probably already have an
item in the GNOME or KDE application menus. If not, just use a open a
terminal and type \texttt{stellarium}.


\subsection{Troubleshooting}
\label{sec:GettingStarted:Running:Troubleshooting}

Stellarium writes startup and other diagnostic messages into a
logfile. Please see section~\ref{sec:FilesAndDirectories} where this
file is located on your system. This file is \emph{essential} in case when
you feel you need to report a problem with your system which has not
been found before.

If you don't succeed in running Stellarium, please see the online
forum\footnote{\url{https://launchpad.net/stellarium}}.  It includes
FAQ (Frequently Asked Questions)\index{FAQ} and a general question
section which may include further hints. Please make sure you have
read and understood the FAQ before asking the same questions again.


%%% Local Variables: 
%%% mode: latex
%%% TeX-PDF-mode: t
%%% TeX-master: "guide"
%%% End: 
