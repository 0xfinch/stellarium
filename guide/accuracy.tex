%% Stellarium user guide.
%% State: 2015-12 wiki->Guide. Was called Precision, but this is wrong.
%% 2016-04 GZ Typofixes, made proper chapter from this. 
%% TODO: (GZ) I am not sure about the 1-arcsecond accuracy of VSOP! More should be written with due diligence.

%\chapterimage{chapter-t1-bg} % Chapter heading image


\chapter{Accuracy}
\label{ch:Accuracy}
Stellarium uses the VSOP87
method\footnote{\url{http://vizier.cfa.harvard.edu/viz-bin/ftp-index?/ftp/cats/VI/81}}
to calculate the positions of the planets over time.

As with other methods, the accuracy of the calculations vary according
to the planet and the time for which one makes the calculation. Reasons
for these inaccuracies include the fact that the motion of the planet
isn't as predictable as Newtonian mechanics would have us believe.

As far as Stellarium is concerned, the user should bear in mind the
following properties of the VSOP87 method. Accuracy values here are
positional as observed from Earth.

\begin{longtabu} to \textwidth {X|l|X}
\toprule
\emph{Object(s)} & \emph{Method} & \emph{Notes}\tabularnewline
\midrule
Mercury, Venus, Earth-Moon barycenter, Mars & VSOP87 & Accuracy is 1 arc-second from 2000 B.C. -- 6000 A.D.\\
\midrule
Jupiter, Saturn                             & VSOP87 & Accuracy is 1 arc-second from 0 A.D. -- 4000 A.D.\\
\midrule
Uranus, Neptune                             & VSOP87 & Accuracy is 1 arc-second from 4000 B.C. -- 8000 A.D.\\
\midrule
Pluto                                       & ?      & Pluto's position is valid only from 1885 A.D. -- 2099 A.D.\\
\midrule
Earth's Moon                                & ELP2000-82B & Unsure about interval of validity or accuracy at time of writing. Possibly valid from 1828 A.D. to 2047 A.D.\\
\midrule
Galilean satellites                         & L2     & Valid from 500 A.D. -- 3500 A.D.\\
\bottomrule
\end{longtabu}




%%% Local Variables: 
%%% mode: latex
%%% TeX-PDF-mode: t
%%% TeX-master: "guide"
%%% End: 

